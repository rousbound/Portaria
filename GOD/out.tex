\documentclass[10pt]{article}
\usepackage[portuguese]{babel}
\usepackage[utf8]{inputenc}
\usepackage[pdftex]{graphicx}
\usepackage{venndiagram}
\usepackage{subcaption}
\usepackage{caption}
\usepackage[backend=biber,style=authoryear-ibid]{biblatex}
\usepackage[normalem]{ulem}
\usepackage[margin=1.8in]{geometry}
%\addbibresource{}
\graphicspath{{Pictures/}}
\usepackage{tikz}
\usepackage{setspace}
\usepackage{enumitem}
\usepackage{textcomp}
\usepackage{float}

\renewcommand{\thesection}{\Roman{section}} 

\author{}
\title{\vspace{-3.5cm}GOD}
\date{}

\newcommand{\mychar}[1]{
  \bigskip
  \hspace{-2em} \MakeUppercase{#1}
}

\begin{document}
\maketitle



\mychar{DEUS}: Tá perdida?. — Tô um pouco.. — Você morreu.. Quê?!. Desencarnou, veio parar aqui.

\mychar{MULHER}: Você é quem?. Deus.. Eu sou Deus, Deus, Deus!. Owaha!. Como assim voc... Você é Deus?. Assim, sendo assim. — Ó, toda civilização acredita em alguma coisa, não é?. — Humrum. —Alguma tinha que tá certa, correto?. — Humrum. E não é que esse tempo todo quem tava certo era o pessoal da tribo da Polinésia?. Caral.... Você, como não seguiu à risca, nossos dogmas, as escrituras linguísticas. Você vai arder, no infinito.. Ô, mas eu não sabia. Eu não sabia que.... Como é que eu ia saber que o deus polinésio era o deus certo?. Você não ia saber

\mychar{DEUS}: Catholic, errou, errou feio errou feio errou rude. — Não tem algum jeito assim, assim pra me redimir?. De acordo com a doutrina se você dançar esfregando  o peito e a barriga no chão, cê se redime. Tá certo... Opa, vamo lá. Ahannn. Ahahaha. Ahahah! Cê acreditou, ô, cegonha!


\end{document}

