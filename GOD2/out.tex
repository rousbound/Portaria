\documentclass[10pt]{article}
\usepackage[portuguese]{babel}
\usepackage[utf8]{inputenc}
\usepackage[pdftex]{graphicx}
\usepackage{venndiagram}
\usepackage{subcaption}
\usepackage{caption}
\usepackage[backend=biber,style=authoryear-ibid]{biblatex}
\usepackage[normalem]{ulem}
\usepackage[margin=1.8in]{geometry}
%\addbibresource{}
\graphicspath{{Pictures/}}
\usepackage{tikz}
\usepackage{setspace}
\usepackage{enumitem}
\usepackage{textcomp}
\usepackage{float}

\renewcommand{\thesection}{\Roman{section}} 

\author{}
\title{\vspace{-3.5cm}GOD2}
\date{}

\newcommand{\mychar}[1]{
  \bigskip
  \hspace{-2em} \MakeUppercase{#1}
}

\begin{document}
\maketitle



\mychar{DEUS}: Kind: captions. — Ó, toda civilização acredita em alguma coisa, não é?. — Humrum. —Alguma tinha que tá certa, correto?. — Humrum

\mychar{MULHER}: Como é que eu ia saber que o deus polinésio era o deus certo?. Você não ia saber. cê não ia saber. Você escolheu o Deus... Deixa eu vê aqui, Judite.... Catholic, errou, errou feio errou feio errou rude. — Não tem algum jeito assim, assim pra me redimir?. De acordo com a doutrina se você dançar esfregando  o peito e a barriga no chão, cê se redime. Opa, vamo lá. Ahannn. Ahahaha

\mychar{DEUS}: Você acredita que eu fiz isso com Madre Tereza de Caucutá?. E ela debatia, babava. Então quer dizer, que eu fui a missa. todo domingo, eu não traí meu marido. eu dei meu dinheiro para os pobres


\end{document}

