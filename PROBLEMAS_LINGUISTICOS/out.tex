\documentclass[10pt]{article}
\usepackage[portuguese]{babel}
\usepackage[utf8]{inputenc}
\usepackage[pdftex]{graphicx}
\usepackage{venndiagram}
\usepackage{subcaption}
\usepackage{caption}
\usepackage[backend=biber,style=authoryear-ibid]{biblatex}
\usepackage[normalem]{ulem}
\usepackage[margin=1.8in]{geometry}
%\addbibresource{}
\graphicspath{{Pictures/}}
\usepackage{tikz}
\usepackage{setspace}
\usepackage{enumitem}
\usepackage{textcomp}
\usepackage{float}

\renewcommand{\thesection}{\Roman{section}} 

\author{}
\title{\vspace{-3.5cm}PROBLEMAS LINGUISTICOS}
\date{}

\newcommand{\mychar}[1]{
  \bigskip
  \hspace{-2em} \MakeUppercase{#1}
}

\begin{document}
\maketitle



\mychar{APRESENTADOR}: Hoje no programa problemas linguísticos.... Eu vou conversar com Marcos.. O Marcos tem um problema com conectivos... Não é isso Marcos?

\mychar{MARCOS}: Apesar que... sim

\mychar{APRESENTADOR}: E quando começou??

\mychar{MARCOS}: Daqui que eu era pequeno. Por isso ultimamente piorou. Se bem foi por causa das drogas, assim das quais eu tomei... em doses cavalares. por causa de manhã cedo

\mychar{APRESENTADOR}: Entendi...

\mychar{MARCOS}: Então você tomou muitas drogas... foi isso Marcos???. Apesar que sim... por causa que eram poucas assim, as quais eram leves

\mychar{APRESENTADOR}: Entendi.... Você esté fazendo algum tipo de tratamento?

\mychar{MARCOS}: Tô... sobretudo falto muito ás sessões. Porém.... São em Copacabana. Mas pego muito trânsito entretanto. Muito mesmo.. Na Barata Ribeiro. Outro sim pego. Ahh, Toneleiro. Por isso. Eu chego em meia hora... desde a Barra. No que se refere a Copacabana

\mychar{APRESENTADOR}: Ok!. Marcos quem tá com a gente aqui também, é a Lucia. A Lucia tem um outro tipo de problema não é, Lucia?

\mychar{LUCIA}: É verdade?

\mychar{APRESENTADOR}: Eu acho que sim

\mychar{LUCIA}: Eu sei que é verdade?

\mychar{APRESENTADOR}: Ah, o problema da Lúcia é com inflexões

\mychar{LUCIA}: Isso?

\mychar{APRESENTADOR}: Ela troca a entonação de perguntas e respostas, não é isso Lúcia?

\mychar{LUCIA}: Agora acertou?

\mychar{APRESENTADOR}: Agora eu acertei sim, cê tá afirmando, sim eu acertei!

\mychar{LUCIA}: Você acha isso grave!

\mychar{APRESENTADOR}: Não sei, eu que te pergunto, você acha isso grave?

\mychar{LUCIA}: Não Acho?

\mychar{APRESENTADOR}: Agora fiquei um pouco na dúvida se você está me perguntando ou respondendo, porque as vezes confunde um pouquinho.

\mychar{LUCIA}: Você pode ajudar a gente!

\mychar{MARCOS}: Quero, apesar que Posso

\mychar{APRESENTADOR}: Bom, a gente vai para um rápido intervalo, resolver este enigma aqui. E no próximo bloco,. Joel Santana e o plural. Fique com a gente!


\end{document}

